\section{Introduction}
\label{sec:intro}

\jr{Microservices are important. }

As the size and complexity of modern, large-scale software applications continue to grow, an increasing number of organizations are adopting architectural strategies that better support the new scalability, deployability, and performance demands of their software systems. The microservice-based approach -- an architectural stretegy that involves dividing applications into separate, independently-deployable components -- is increasingly being used as a solution to these growing needs. 


\jr{Splitting monolith to microservies is challenging. }

Despite the technique's growing popularity, there is a distinct lack of ubiquitous, standardized procudures to reference when splitting a monolithic application into microservices. Most proponents of this architectural stule cite domain-driven and business capability-centered design as optimal approaches to monolithic decomposition \lk{INSERT CITATIONS HERE} \lk{INSERT INFO ABOUT WHAT THESE TECHNIQUES ARE ARE WHY THEY ARE CONSIDERED "OPTIMAL"}; however, such approaches often require significant familiarity with the structure and inner-workings of both the software system in question and the organization that manages it. In addition, for particularly large applications, the decomposition process entails manually sorting hundreds, if not thousands, of files, classes, and/or methods into groups -- a process that requires significant time and effort whilst being prone to the subjective architectural preferences of whomever is performing the decomposition. 

\jr{Tools exist. Based on static and dynamic analysis. Discuss the theoretical differences between static and dynamic analysis. }

\jr{Select one static and one dynamic tool. Describe the tools.} 

\jr{Compare under the common setup. Discuss the setup and evaluation subjects.} 

\jr{Results of the analysis.} 

\jr{Suggestions for future research.}

The remainder of the paper is structured as follows. Section~\ref{sec:background}  \jr{TBD}. 

