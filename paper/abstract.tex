\begin{abstract}
The microservice-based architecture -- a design principle wherein an application is divided into a suite of independently deployable services -- has been frequently adopted by organizations looking to expand the scalability, deployability, and maintainability of their software systems. Despite the architecture's recent popularity, there is a lack of systematically-evaluated tools that automatically refactor monolithic applications into high-quality microservice-based systems. 
In this study, we select two most prominent approaches to microservice extraction -- one that is based on static and another that 
is based on dynamic analysis. We run these approaches on real-world monolithic applications, and evaluate the results against microservice-based representations of these applications produced manually by third-party experts. We discuss the accuracy and limitations of the considered approaches and provide suggestions for future research.
\end{abstract}